\documentclass[12pt]{article}
\usepackage{ctex} % 直接使用ctex包支持中文
\usepackage{amsmath, amssymb, chemformula}
\usepackage{mhchem}

\title{quiz11}
\author{}
\date{}

\begin{document}
	
	\maketitle
	
	\section*{解答}
	
	\subsection*{1}
	
	1. 在第四周期过渡金属 (\(\ce{Sc, Ti, V, Cr, Mn, Fe, Co, Ni, Cu, Zn}\)) 中,\(\ce{Cu}\) 和 \(\ce{Zn}\) 的熔点与沸点相对较低。
	
	2. 电子组态(基态原子态):
	\[
	\ce{Cu}:\; [\ce{Ar}]\,3d^{10}4s^1
	\]
	\[
	\ce{Zn}:\; [\ce{Ar}]\,3d^{10}4s^2
	\]
	
	3. 解释:当过渡金属的 \(\mathrm{d}\)-轨道充满(如 \(\ce{Zn}\) 为 \(3d^{10}\)),或接近满轨道(如 \(\ce{Cu}\) 的 \(3d^{10}\)),\(\mathrm{d}\) 电子对金属-金属键的贡献减弱。此时金属键相对较弱,使晶格能降低,从而导致较低的熔点与沸点。
	
	4. 汞(\(\ce{Hg}\))在室温为液态的原因:  
	汞的 6s 轨道电子受相对论效应影响,轨道收缩明显,使得汞原子间的金属键更弱。因此,汞在常温下即为液态。
	
	\subsection*{配合物金属中心的价电子数计算}
	
	\subsubsection*{(a) \(\ce{K3[Fe(CN)6]}\)}
	
	- \(\ce{CN^-}\) 为 -1 配体,\(\ce{[Fe(CN)6]^3-}\) 说明配阴离子总体电荷为 -3。  
	已知:\(\ce{K+}\) 为 +1,共 3 个 \(\ce{K+}\) 中和 \(\ce{[Fe(CN)6]^3-}\)。
	
	- 确定 \(\ce{Fe}\) 氧化态:  
	令 \(\ce{Fe}\) 的氧化态为 \(x\),\(\ce{CN^-}\) 为 -1,有 6 个:  
	\[
	x + 6(-1) = -3 \implies x - 6 = -3 \implies x = +3
	\]
	因此 \(\ce{Fe}\) 为 \(\ce{Fe(III)}\)。
	
	- \(\ce{Fe}\) 基态为 \([ \ce{Ar}]\,3d^6\,4s^2\)。  
	去掉 3 个电子(先从 4s,再从 3d)以形成 \(\ce{Fe^{3+}}\):  
	\[
	\ce{Fe^{3+}}: [\ce{Ar}]\,3d^5
	\]
	因此 \(\ce{Fe(III)}\) 的 d 电子数为 5,无 s、p 电子残留。  
	故 (d + s + p) = 5。
	
	\subsubsection*{(b) \(\ce{[Co(NH3)5Cl]Cl2}\)}
	
	- 外部有 2 个 \(\ce{Cl^-}\),总电荷为 -2。整体分子中性,则内配离子电荷为 +2:
	\[
	[\ce{Co(NH3)5Cl}]^{2+}
	\]
	
	- 配离子中,\(\ce{NH3}\) 为中性,\(\ce{Cl^-}\) 为 -1:  
	\[
	x + 5(0) + (-1) = +2 \implies x - 1 = +2 \implies x = +3
	\]
	因此 \(\ce{Co}\) 为 \(\ce{Co(III)}\)。
	
	- \(\ce{Co}\) 基态为 \([ \ce{Ar}]\,3d^7\,4s^2\)。  
	去除 3 个电子形成 \(\ce{Co^{3+}}\) (先去 4s²,再去 3d¹):  
	\[
	\ce{Co^{3+}}: [\ce{Ar}]\,3d^6
	\]
	故 \(\ce{Co(III)}\) 的 d 电子数为 6,无 s、p 电子残留。  
	(d + s + p) = 6。
	
	
	
\end{document}
